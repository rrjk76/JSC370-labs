% Options for packages loaded elsewhere
\PassOptionsToPackage{unicode}{hyperref}
\PassOptionsToPackage{hyphens}{url}
%
\documentclass[
]{article}
\usepackage{amsmath,amssymb}
\usepackage{iftex}
\ifPDFTeX
  \usepackage[T1]{fontenc}
  \usepackage[utf8]{inputenc}
  \usepackage{textcomp} % provide euro and other symbols
\else % if luatex or xetex
  \usepackage{unicode-math} % this also loads fontspec
  \defaultfontfeatures{Scale=MatchLowercase}
  \defaultfontfeatures[\rmfamily]{Ligatures=TeX,Scale=1}
\fi
\usepackage{lmodern}
\ifPDFTeX\else
  % xetex/luatex font selection
\fi
% Use upquote if available, for straight quotes in verbatim environments
\IfFileExists{upquote.sty}{\usepackage{upquote}}{}
\IfFileExists{microtype.sty}{% use microtype if available
  \usepackage[]{microtype}
  \UseMicrotypeSet[protrusion]{basicmath} % disable protrusion for tt fonts
}{}
\makeatletter
\@ifundefined{KOMAClassName}{% if non-KOMA class
  \IfFileExists{parskip.sty}{%
    \usepackage{parskip}
  }{% else
    \setlength{\parindent}{0pt}
    \setlength{\parskip}{6pt plus 2pt minus 1pt}}
}{% if KOMA class
  \KOMAoptions{parskip=half}}
\makeatother
\usepackage{xcolor}
\usepackage[margin=1in]{geometry}
\usepackage{color}
\usepackage{fancyvrb}
\newcommand{\VerbBar}{|}
\newcommand{\VERB}{\Verb[commandchars=\\\{\}]}
\DefineVerbatimEnvironment{Highlighting}{Verbatim}{commandchars=\\\{\}}
% Add ',fontsize=\small' for more characters per line
\usepackage{framed}
\definecolor{shadecolor}{RGB}{248,248,248}
\newenvironment{Shaded}{\begin{snugshade}}{\end{snugshade}}
\newcommand{\AlertTok}[1]{\textcolor[rgb]{0.94,0.16,0.16}{#1}}
\newcommand{\AnnotationTok}[1]{\textcolor[rgb]{0.56,0.35,0.01}{\textbf{\textit{#1}}}}
\newcommand{\AttributeTok}[1]{\textcolor[rgb]{0.13,0.29,0.53}{#1}}
\newcommand{\BaseNTok}[1]{\textcolor[rgb]{0.00,0.00,0.81}{#1}}
\newcommand{\BuiltInTok}[1]{#1}
\newcommand{\CharTok}[1]{\textcolor[rgb]{0.31,0.60,0.02}{#1}}
\newcommand{\CommentTok}[1]{\textcolor[rgb]{0.56,0.35,0.01}{\textit{#1}}}
\newcommand{\CommentVarTok}[1]{\textcolor[rgb]{0.56,0.35,0.01}{\textbf{\textit{#1}}}}
\newcommand{\ConstantTok}[1]{\textcolor[rgb]{0.56,0.35,0.01}{#1}}
\newcommand{\ControlFlowTok}[1]{\textcolor[rgb]{0.13,0.29,0.53}{\textbf{#1}}}
\newcommand{\DataTypeTok}[1]{\textcolor[rgb]{0.13,0.29,0.53}{#1}}
\newcommand{\DecValTok}[1]{\textcolor[rgb]{0.00,0.00,0.81}{#1}}
\newcommand{\DocumentationTok}[1]{\textcolor[rgb]{0.56,0.35,0.01}{\textbf{\textit{#1}}}}
\newcommand{\ErrorTok}[1]{\textcolor[rgb]{0.64,0.00,0.00}{\textbf{#1}}}
\newcommand{\ExtensionTok}[1]{#1}
\newcommand{\FloatTok}[1]{\textcolor[rgb]{0.00,0.00,0.81}{#1}}
\newcommand{\FunctionTok}[1]{\textcolor[rgb]{0.13,0.29,0.53}{\textbf{#1}}}
\newcommand{\ImportTok}[1]{#1}
\newcommand{\InformationTok}[1]{\textcolor[rgb]{0.56,0.35,0.01}{\textbf{\textit{#1}}}}
\newcommand{\KeywordTok}[1]{\textcolor[rgb]{0.13,0.29,0.53}{\textbf{#1}}}
\newcommand{\NormalTok}[1]{#1}
\newcommand{\OperatorTok}[1]{\textcolor[rgb]{0.81,0.36,0.00}{\textbf{#1}}}
\newcommand{\OtherTok}[1]{\textcolor[rgb]{0.56,0.35,0.01}{#1}}
\newcommand{\PreprocessorTok}[1]{\textcolor[rgb]{0.56,0.35,0.01}{\textit{#1}}}
\newcommand{\RegionMarkerTok}[1]{#1}
\newcommand{\SpecialCharTok}[1]{\textcolor[rgb]{0.81,0.36,0.00}{\textbf{#1}}}
\newcommand{\SpecialStringTok}[1]{\textcolor[rgb]{0.31,0.60,0.02}{#1}}
\newcommand{\StringTok}[1]{\textcolor[rgb]{0.31,0.60,0.02}{#1}}
\newcommand{\VariableTok}[1]{\textcolor[rgb]{0.00,0.00,0.00}{#1}}
\newcommand{\VerbatimStringTok}[1]{\textcolor[rgb]{0.31,0.60,0.02}{#1}}
\newcommand{\WarningTok}[1]{\textcolor[rgb]{0.56,0.35,0.01}{\textbf{\textit{#1}}}}
\usepackage{graphicx}
\makeatletter
\def\maxwidth{\ifdim\Gin@nat@width>\linewidth\linewidth\else\Gin@nat@width\fi}
\def\maxheight{\ifdim\Gin@nat@height>\textheight\textheight\else\Gin@nat@height\fi}
\makeatother
% Scale images if necessary, so that they will not overflow the page
% margins by default, and it is still possible to overwrite the defaults
% using explicit options in \includegraphics[width, height, ...]{}
\setkeys{Gin}{width=\maxwidth,height=\maxheight,keepaspectratio}
% Set default figure placement to htbp
\makeatletter
\def\fps@figure{htbp}
\makeatother
\setlength{\emergencystretch}{3em} % prevent overfull lines
\providecommand{\tightlist}{%
  \setlength{\itemsep}{0pt}\setlength{\parskip}{0pt}}
\setcounter{secnumdepth}{-\maxdimen} % remove section numbering
\ifLuaTeX
  \usepackage{selnolig}  % disable illegal ligatures
\fi
\usepackage{bookmark}
\IfFileExists{xurl.sty}{\usepackage{xurl}}{} % add URL line breaks if available
\urlstyle{same}
\hypersetup{
  pdftitle={Lab 9 - HPC},
  hidelinks,
  pdfcreator={LaTeX via pandoc}}

\title{Lab 9 - HPC}
\author{}
\date{\vspace{-2.5em}}

\begin{document}
\maketitle

\section{Learning goals}\label{learning-goals}

In this lab, you are expected to practice the following skills:

\begin{itemize}
\tightlist
\item
  Evaluate whether a problem can be parallelized or not.
\item
  Practice with the parallel package.
\item
  Use Rscript to submit jobs.
\end{itemize}

\subsection{Problem 1}\label{problem-1}

Give yourself a few minutes to think about what you learned about
parallelization. List three examples of problems that you believe may be
solved using parallel computing, and check for packages on the HPC CRAN
task view that may be related to it.

\begin{itemize}
\item
  Cross-Validation in machine learning - distributing training
\item
  caret package -\textgreater{} supports parallel cross-validation with
  \texttt{doParallel}
\item
  mlr, foreach, doParallel -\textgreater{} for parallel model training
\item
  Bootstrapping (like for estimating confidence intervals, drawing
  random samples)
\item
  boot -\textgreater{} for bootstrapping
\item
  parallel -\textgreater{} parallelize resampling
\item
  doParallel, foreach
\item
  Bayesian Inference:
\item
  markove chain monte carlo (when approximating distributions and
  generating samples from the posterior distributions)\\
\item
  parallel -rstan -\textgreater{} for stan in bayesian modelling
\item
  RcppParallel -\textgreater{} parallel mcmc sampling automatically
  -nimle -\textgreater{} customize bayesian inference
\end{itemize}

\subsection{Problem 2:
Pre-parallelization}\label{problem-2-pre-parallelization}

The following functions can be written to be more efficient without
using \texttt{parallel}:

\begin{enumerate}
\def\labelenumi{\arabic{enumi}.}
\tightlist
\item
  This function generates a \texttt{n\ x\ k} dataset with all its
  entries having a Poisson distribution with mean \texttt{lambda}.
\end{enumerate}

\begin{Shaded}
\begin{Highlighting}[]
\CommentTok{\# inefficient because it\textquotesingle{}s using a for loop with rbind (reallocating memory each time). this wastes a lot of computation power when reallocating memory (worse with larger matrices). }
\CommentTok{\# using rbind is very slow, because R copies every object at every iteration }
\NormalTok{fun1 }\OtherTok{\textless{}{-}} \ControlFlowTok{function}\NormalTok{(}\AttributeTok{n =} \DecValTok{100}\NormalTok{, }\AttributeTok{k =} \DecValTok{4}\NormalTok{, }\AttributeTok{lambda =} \DecValTok{4}\NormalTok{) \{}
\NormalTok{  x }\OtherTok{\textless{}{-}} \ConstantTok{NULL}
  
  \ControlFlowTok{for}\NormalTok{ (i }\ControlFlowTok{in} \DecValTok{1}\SpecialCharTok{:}\NormalTok{n)}
\NormalTok{    x }\OtherTok{\textless{}{-}} \FunctionTok{rbind}\NormalTok{(x, }\FunctionTok{rpois}\NormalTok{(k, lambda))}
  
  \FunctionTok{return}\NormalTok{(x)}
\NormalTok{\}}

\NormalTok{fun1alt }\OtherTok{\textless{}{-}} \ControlFlowTok{function}\NormalTok{(}\AttributeTok{n =} \DecValTok{100}\NormalTok{, }\AttributeTok{k =} \DecValTok{4}\NormalTok{, }\AttributeTok{lambda =} \DecValTok{4}\NormalTok{) \{}
  \CommentTok{\# YOUR CODE HERE}
  \FunctionTok{matrix}\NormalTok{(}\FunctionTok{rpois}\NormalTok{(n}\SpecialCharTok{*}\NormalTok{k, }\AttributeTok{lambda =}\NormalTok{ lambda), }\AttributeTok{ncol =}\NormalTok{ k)}
  \CommentTok{\# here, we\textquotesingle{}re using matrix function instead of for loop and rbind}
  \CommentTok{\# matrix allows us to pre{-}allocate memory. }
  \CommentTok{\# rpois(n*k, lambda = lambda) allows us to generate the random numbers all at once }
\NormalTok{\}}

\CommentTok{\# Benchmarking}
\NormalTok{microbenchmark}\SpecialCharTok{::}\FunctionTok{microbenchmark}\NormalTok{(}
  \FunctionTok{fun1}\NormalTok{(}\DecValTok{100}\NormalTok{),}
  \FunctionTok{fun1alt}\NormalTok{(}\DecValTok{100}\NormalTok{),}
  \AttributeTok{unit =} \StringTok{"ns"}
\NormalTok{)}
\end{Highlighting}
\end{Shaded}

\begin{verbatim}
## Warning in microbenchmark::microbenchmark(fun1(100), fun1alt(100), unit =
## "ns"): less accurate nanosecond times to avoid potential integer overflows
\end{verbatim}

\begin{verbatim}
## Unit: nanoseconds
##          expr    min       lq      mean   median       uq     max neval
##     fun1(100) 124722 133639.5 153851.68 140650.5 144012.5 1535614   100
##  fun1alt(100)  13120  13919.5  87048.33  14432.0  15252.0 7184799   100
\end{verbatim}

How much faster? \emph{Much faster in every term, implying that the new
function is more efficient. After some calculations, I found that
fun1alt(100) is approximately 8.88 times faster than fun1(100).}

\begin{enumerate}
\def\labelenumi{\arabic{enumi}.}
\setcounter{enumi}{1}
\tightlist
\item
  Find the column max (hint: Checkout the function \texttt{max.col()}).
\end{enumerate}

\begin{Shaded}
\begin{Highlighting}[]
\CommentTok{\# Data Generating Process (10 x 10,000 matrix)}
\FunctionTok{set.seed}\NormalTok{(}\DecValTok{1234}\NormalTok{)}
\NormalTok{x }\OtherTok{\textless{}{-}} \FunctionTok{matrix}\NormalTok{(}\FunctionTok{rnorm}\NormalTok{(}\FloatTok{1e4}\NormalTok{), }\AttributeTok{nrow=}\DecValTok{10}\NormalTok{)}

\CommentTok{\# Find each column\textquotesingle{}s max value}
\NormalTok{fun2 }\OtherTok{\textless{}{-}} \ControlFlowTok{function}\NormalTok{(x) \{}
  \FunctionTok{apply}\NormalTok{(x, }\DecValTok{2}\NormalTok{, max)}
\NormalTok{\}}

\NormalTok{fun2alt }\OtherTok{\textless{}{-}} \ControlFlowTok{function}\NormalTok{(x) \{}
  \CommentTok{\# YOUR CODE HERE}
\NormalTok{  x[}\FunctionTok{cbind}\NormalTok{(}\FunctionTok{max.col}\NormalTok{(}\FunctionTok{t}\NormalTok{(x)), }\DecValTok{1}\SpecialCharTok{:}\FunctionTok{ncol}\NormalTok{(x))]}
  \CommentTok{\# avoids function calls inside loops}
  \CommentTok{\# directly extracts the max values }
\NormalTok{\}}

\CommentTok{\# Benchmarking}
\NormalTok{bench }\OtherTok{\textless{}{-}}\NormalTok{ microbenchmark}\SpecialCharTok{::}\FunctionTok{microbenchmark}\NormalTok{(}
  \FunctionTok{fun2}\NormalTok{(x),}
  \FunctionTok{fun2alt}\NormalTok{(x),}
  \AttributeTok{unit =} \StringTok{"micro"}
\NormalTok{)}
\end{Highlighting}
\end{Shaded}

\begin{itemize}
\tightlist
\item
  inefficient, b/c it loops through all columns and applies the max
  function, which leads to a slower run time. (apply function is slower
  than direct matrix indexing) \emph{Answer here with a plot.}
\end{itemize}

\begin{Shaded}
\begin{Highlighting}[]
\CommentTok{\# plot }
\FunctionTok{plot}\NormalTok{(bench)}
\end{Highlighting}
\end{Shaded}

\includegraphics{lab09-hpc_files/figure-latex/unnamed-chunk-2-1.pdf}

\begin{Shaded}
\begin{Highlighting}[]
\NormalTok{ggplot2}\SpecialCharTok{::} \FunctionTok{autoplot}\NormalTok{(bench) }\SpecialCharTok{+} 
\NormalTok{  ggplot2}\SpecialCharTok{::}\FunctionTok{theme\_bw}\NormalTok{()}
\end{Highlighting}
\end{Shaded}

\includegraphics{lab09-hpc_files/figure-latex/unnamed-chunk-2-2.pdf}

\subsection{Problem 3: Parallelize
everything}\label{problem-3-parallelize-everything}

We will now turn our attention to non-parametric
\href{https://en.wikipedia.org/wiki/Bootstrapping_(statistics)}{bootstrapping}.
Among its many uses, non-parametric bootstrapping allow us to obtain
confidence intervals for parameter estimates without relying on
parametric assumptions.

The main assumption is that we can approximate many experiments by
resampling observations from our original dataset, which reflects the
population.

This function implements the non-parametric bootstrap:

\begin{Shaded}
\begin{Highlighting}[]
\FunctionTok{library}\NormalTok{(parallel)}
\NormalTok{my\_boot }\OtherTok{\textless{}{-}} \ControlFlowTok{function}\NormalTok{(dat, stat, R, }\AttributeTok{ncpus =} \DecValTok{1}\DataTypeTok{L}\NormalTok{) \{}
  
  \CommentTok{\# Getting the random indices}
\NormalTok{  n }\OtherTok{\textless{}{-}} \FunctionTok{nrow}\NormalTok{(dat)}
\NormalTok{  idx }\OtherTok{\textless{}{-}} \FunctionTok{matrix}\NormalTok{(}\FunctionTok{sample.int}\NormalTok{(n, n}\SpecialCharTok{*}\NormalTok{R, }\ConstantTok{TRUE}\NormalTok{), }\AttributeTok{nrow=}\NormalTok{n, }\AttributeTok{ncol=}\NormalTok{R)}
 
  \CommentTok{\# Making the cluster using \textasciigrave{}ncpus\textasciigrave{}}
  \CommentTok{\# STEP 1: GOES HERE}
  \CommentTok{\# creates a cluster for parallel computer: }
  \CommentTok{\# ncpus specifies using multiple CPU cores }
  \CommentTok{\# PSOCK = parallel socket cluster }
  \CommentTok{\# need a cluster, b/c R by default runs sequentially. By creating clusters, it allows us     to run multiple independent computations, running many across the CPU ports }
\NormalTok{  cl }\OtherTok{\textless{}{-}} \FunctionTok{makePSOCKcluster}\NormalTok{(ncpus)}
  
  \CommentTok{\# STEP 2: GOES HERE}
  \CommentTok{\# on.exit(stopCluster(cl)) automatically shuts down clusters. w/o this, the clusters        continue, takes up memory, and could cause memory leaks }
  \CommentTok{\# export the variables to the cluster }
  
  \FunctionTok{clusterExport}\NormalTok{(cl, }\AttributeTok{varlist =} \FunctionTok{c}\NormalTok{(}\StringTok{"idx"}\NormalTok{, }\StringTok{"dat"}\NormalTok{, }\StringTok{"stat"}\NormalTok{), }\AttributeTok{envir =} \FunctionTok{environment}\NormalTok{()) }
    \CommentTok{\# sending variables to all worker nodes}
    \CommentTok{\# each run in isolated environments, don\textquotesingle{}t have access to global variable }
    \CommentTok{\# idx {-}\textgreater{} resampling indices for bootstrap }
    \CommentTok{\# dat {-}\textgreater{} dataset }
    \CommentTok{\# stat {-}\textgreater{} the statistical function that we use to compute the estimates }
  
  
  \CommentTok{\# change sequential apply to parallelized apply}
  \CommentTok{\# STEP 3: THIS FUNCTION NEEDS TO BE REPLACED WITH parLapply}
\NormalTok{  ans }\OtherTok{\textless{}{-}} \FunctionTok{parLapply}\NormalTok{(cl, }\FunctionTok{seq\_len}\NormalTok{(R), }\ControlFlowTok{function}\NormalTok{(i) \{}
    \FunctionTok{stat}\NormalTok{(dat[idx[,i], , }\AttributeTok{drop=}\ConstantTok{FALSE}\NormalTok{])}
\NormalTok{  \})}
  
  \CommentTok{\# Coercing the list into a matrix}
\NormalTok{  ans }\OtherTok{\textless{}{-}} \FunctionTok{do.call}\NormalTok{(rbind, ans)}
  
  \CommentTok{\# STEP 4: GOES HERE}
  \FunctionTok{stopCluster}\NormalTok{(cl)}
  \CommentTok{\# why? }
    \CommentTok{\# to free up the system resources }
  
\NormalTok{  ans}
  
\NormalTok{\}}
\end{Highlighting}
\end{Shaded}

\begin{enumerate}
\def\labelenumi{\arabic{enumi}.}
\tightlist
\item
  Use the previous pseudocode, and make it work with \texttt{parallel}.
  Here is just an example for you to try:
\end{enumerate}

\begin{Shaded}
\begin{Highlighting}[]
\CommentTok{\# Bootstrap of a linear regression model}
\NormalTok{my\_stat }\OtherTok{\textless{}{-}} \ControlFlowTok{function}\NormalTok{(d)  }\FunctionTok{coef}\NormalTok{(}\FunctionTok{lm}\NormalTok{(y}\SpecialCharTok{\textasciitilde{}}\NormalTok{x, }\AttributeTok{data =}\NormalTok{ d))}

\CommentTok{\# DATA SIM}
\FunctionTok{set.seed}\NormalTok{(}\DecValTok{1}\NormalTok{)}
\NormalTok{n }\OtherTok{\textless{}{-}} \DecValTok{500} 
\NormalTok{R }\OtherTok{\textless{}{-}} \FloatTok{1e4}
\NormalTok{x }\OtherTok{\textless{}{-}} \FunctionTok{cbind}\NormalTok{(}\FunctionTok{rnorm}\NormalTok{(n)) }
\NormalTok{y }\OtherTok{\textless{}{-}}\NormalTok{ x}\SpecialCharTok{*}\DecValTok{5} \SpecialCharTok{+} \FunctionTok{rnorm}\NormalTok{(n)}

\CommentTok{\# Check if we get something similar as lm}
\CommentTok{\# OLS CI }
\NormalTok{ans0 }\OtherTok{\textless{}{-}} \FunctionTok{confint}\NormalTok{(}\FunctionTok{lm}\NormalTok{(y}\SpecialCharTok{\textasciitilde{}}\NormalTok{x))}
\FunctionTok{cat}\NormalTok{(}\StringTok{"OLS CI }\SpecialCharTok{\textbackslash{}n}\StringTok{"}\NormalTok{)}
\FunctionTok{print}\NormalTok{(ans0)}

\NormalTok{ans1 }\OtherTok{\textless{}{-}} \FunctionTok{my\_boot}\NormalTok{(}\AttributeTok{dat =} \FunctionTok{data.frame}\NormalTok{(x, y), my\_stat, }\AttributeTok{R =}\NormalTok{ R, }\AttributeTok{ncpus =} \DecValTok{4}\NormalTok{)}
\NormalTok{qs }\OtherTok{\textless{}{-}} \FunctionTok{c}\NormalTok{(.}\DecValTok{025}\NormalTok{, .}\DecValTok{975}\NormalTok{)}
\FunctionTok{cat}\NormalTok{(}\StringTok{"Bootsrap CI }\SpecialCharTok{\textbackslash{}n}\StringTok{"}\NormalTok{)}
\FunctionTok{print}\NormalTok{(}\FunctionTok{t}\NormalTok{(}\FunctionTok{apply}\NormalTok{(ans1, }\DecValTok{2}\NormalTok{, quantile, }\AttributeTok{probs =}\NormalTok{ qs)))}
\end{Highlighting}
\end{Shaded}

\begin{enumerate}
\def\labelenumi{\arabic{enumi}.}
\setcounter{enumi}{1}
\tightlist
\item
  Check whether your version actually goes faster than the non-parallel
  version:
\end{enumerate}

\begin{Shaded}
\begin{Highlighting}[]
\CommentTok{\# your code here}
\NormalTok{parallel}\SpecialCharTok{::}\FunctionTok{detectCores}\NormalTok{()}

\CommentTok{\# non{-}parallel 1 core}
\FunctionTok{system.time}\NormalTok{(}\FunctionTok{my\_boot}\NormalTok{(}\AttributeTok{dat =} \FunctionTok{data.frame}\NormalTok{(x, y), my\_stat, }\AttributeTok{R =}\NormalTok{ R, }\AttributeTok{ncpus =} \DecValTok{1}\DataTypeTok{L}\NormalTok{))}


\CommentTok{\# parallel 8 cores}
\FunctionTok{system.time}\NormalTok{(}\FunctionTok{my\_boot}\NormalTok{(}\AttributeTok{dat =} \FunctionTok{data.frame}\NormalTok{(x, y), my\_stat, }\AttributeTok{R =}\NormalTok{ R, }\AttributeTok{ncpus =} \DecValTok{8}\DataTypeTok{L}\NormalTok{))}
\end{Highlighting}
\end{Shaded}

\emph{Yes, my version does actually go faster than the non-parallel
version. This is seen, as the 8 core one runs much faster than the 1
core one. As seen above, the elapsed time for the parallel 8 cores is
shorter than the elapsed time for the non-parallel 1 core (3.134
\textgreater{} 1.743)}

\subsection{Problem 4: Compile this markdown document using
Rscript}\label{problem-4-compile-this-markdown-document-using-rscript}

Once you have saved this Rmd file, try running the following command in
your terminal:

\begin{Shaded}
\begin{Highlighting}[]
\ExtensionTok{Rscript} \AttributeTok{{-}{-}vanilla} \AttributeTok{{-}e} \StringTok{\textquotesingle{}rmarkdown::render("[full{-}path{-}to{-}your{-}Rmd{-}file.Rmd]")\textquotesingle{}} \KeywordTok{\&}
\end{Highlighting}
\end{Shaded}

Where \texttt{{[}full-path-to-your-Rmd-file.Rmd{]}} should be replace
with the full path to your Rmd file\ldots{} :).

\end{document}
